\section{The Course}

We call our course ``Autonomous Robotics Lab". It is offered every semester, initially to a maximum of 12 students. As mentioned, in order to provide continuity and a sense of a greater purpose, we choose a ``major project" objective which was compelling to students, could not be solved in one semester, and provided the kind of learning opportunities that we were after. 

The goal we set for ourselves is the creation of an autonomously navigating robot that could pick up and deliver small packages across campus. As it turns out this is in general an unsolved problem with plenty of juicy subproblems to keep us occupied indefinitely.

\subsection{Who is admitted} As an experimental course we are being highly selective of the (up to 12) students who can take this course. Being experimental means a curriculum that is fluid with incomplete scaffolding and teaching staff who is also still learning. We administer this as follows:

We are admitting only Juniors, Seniors and Graduate students. The catalog states that permission from the instructor is required. Students are asked to fill out a short (Google) survey asking about grades received in key Computer Science courses as well as a statement of qualification stating in their own words what previous experiences prepare them for the course. And finally, they are individually interviewed over email to determine that they are strong programmers, disciplined, motivated and with a demonstrated track record.

\subsection{Project and team-based learning} Given that each semester would be different we had to have a clear sense of our learning objectives. In a general sense, the goal is to expose students to the problem solving learning that comes into play when working on a large software project with many moving parts.

\subsection{Learning Objectives} Our objectives fall into two categoies, Robotics and Software Engineering. Here they are, edited for length:

\begin{itemize}
\item Understand fundamentals of mobile autonomous robotics, and the notions of localization, navigation and planning, including being able to explain the SLAM algorithms.
\item Demonstrate the ability to design, implement and test programs written for ROS (Robot Operating System.) Be able to explain as well as implement the key primitives of ROS: nodes, topics, commands and services.
\item Be effective working in teams, designing new algorithms and solving problems of navigation and robotics, brainstorming, collaborating, implementing, testing and demonstrating the results of their work.
\item Demonstrate professional and agile software engineering processes, including writing elegant, readable, documented code, working in rapid iterations, each with a goal and a demo, and performing weekly standup meetings.
\end{itemize}
