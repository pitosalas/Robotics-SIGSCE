\documentclass[sigconf, authordraft]{acmart}

\usepackage{booktabs} % For formal tables

% Copyright
%\setcopyright{none}
%\setcopyright{acmcopyright}
%\setcopyright{acmlicensed}
\setcopyright{rightsretained}
%\setcopyright{usgov}
%\setcopyright{usgovmixed}
%\setcopyright{cagov}
%\setcopyright{cagovmixed}


%Conference
\acmConference[SIGCSE 2019]{SIGCSE 2018}{Feb 2019}{place}
\acmYear{2019}
\copyrightyear{2019}

\begin{document}
\title{A Multi-Semester, Multi-Cohort Course Design: Purpose, Challenges and Results}

\author{R. Pito Salas}
\affiliation{%
  \institution{Brandeis University}
  \streetaddress{200 South Street.}
  \city{Waltham}
  \state{Massachusetts}
  \postcode{12345}
}
\email{pitosalas@brandeis.edu}

\begin{abstract}
This paper explores a novel structure for a course, which we call ``Multi-Semester, Multi-Cohort”. The course structure is framed around a long-term goal or objective, across multiple semesters and with different groups of students. We know that the objective is challenging enough that it will take more than one semester to achieve. As a result, the students working on the project will be different every semester. While this model can be applied in a variety of applied disciplines, the course in this instance is in Robotics. This model has some very interesting benefits but raises unique challenges. Our experience with the ``Autonomous Robotics Lab" covers two completed semesters, and one about to begin. In this paper we will explain our motivation for using this structure, details of the curriculum and pedagogical approach to it including assessment, the type and effectiveness of the scaffolding that was created, and report on our experience including what went well and what we need to improve.
\end{abstract}
\keywords{CS education, course design, projects, teams, robotics}

\section{INTRODUCTION}

As a research university, we offer undergraduate students opportunities to work on major research projects, with different students rotating in and out, semester to semester. For those more interested in the applied disciplines (e.g. engineering, journalism, law, business, social work) there are fewer such opportunities. Our innovation is a course design which would give the ``major project" experience in an experiential form, for students in the applied disciplines. We call it multi-semester/multi-cohort, \textbf{MSMC} in this paper.

\begin{figure}[!ht]
  \centering
  \includegraphics[width=\columnwidth,height=6cm]{diag1}
  \caption{Multi Semester/Multi Cohort structure}
  \label{fig:diag1}
\end{figure}

``Multi-semester'' because the project will not (cannot) be completed in one semester, and ``multi-cohort'' because the set of students working on it changes from semester to semester. This approach has many benefits, but poses some obvious challenges which will be addressed below.

We are inspired by the idea of a \textbf{BHAG} (\textit{Big Hairy Audacious Goal}), conceptualized in \cite{Collins}. Collins and Porras describe this mouthful term as follows:

\begin{quote}
``We found in our research that visionary companies often use bold missions --- BHAGs, shorthand for Big, Hairy, Audacious Goals] --- as a powerful way to stimulate progress.... A true BHAG is clear and compelling, serves as a unifying focal point of effort, and acts as a catalyst for team spirit. It has a clear finish line, so the organization can know when it has achieved the [goal] A BHAG engages people --- it reaches out and grabs them. It is tangible, energizing, highly focused.\cite{Collins}''
\end{quote}

Obviously our ``MSMC'' model for course design is a variant of well established ``Project-Based Learning'' pedagogy\cite{projects}, and our ``BHAG'' just another name for the ``driving question'' defined in \cite{blumenfeld}. We are attracted to idea of a BHAG because it clearly implies a driving question which almost by definition cannot be completed in one semester, thereby supporting the idea of a multi-semester course structure.

For our course, our BHAG is ''Campus Rover'', a robot that can travel both indoors and outdoors on campus to deliver packages from an office in one building to an office in another building. After our first semester pilot, we made changes and refinements in the program, and we are now in our third semester. This paper describes the curriculum and how it has evolved and explores what works and does not.


\section{Review of Relevant Literature}

As far as we know we are the first to use the phrase "Multi-Semester, Multi-Cohort" so there will be no mention of it in the literature. But other than the phrase, there is certainly other relevant work that is quite instructive. 

One highly relevant example comes from ``Learning by Doing: Reflections of the Epics Program \cite{Epics}", Zoltowski and Oakes describe a ``multidisciplinary, vertically-integrated, student led, service-learning design course", where students of all levels join and participate in multi-semester service projects built around a central framework, called ``Epics". Here they describe in more detail how the service projects may span more than one semester or even academic year. This approach is analogous to our Multi-Semester, Multi-Cohort model, except that in our course there is but a single overarching objective. 

``However, in EPICS, the time lines are decoupled, so that projects can extend beyond one academic term. This allows us to scope projects to meet the needs of the community partner, not the requirements of the academic time line. It allows for iteration within the design process as problems and improvements are identified. In addition, students can participate in multiple semesters."

An interesting feature is that the courses count for one half the normal course credits, which allows students to take the course more than once. This could be used to help create continuity from semester to semester, as will be described below. 

Zoltowski and Oakes also mention the challenges of a multi-semester structure. They say that:  ``Multi-semester projects require structures different from most courses to allow transitioning and access to resources from one semester to another..." After which they describe the challenges of sharing reports from one semester to the next. Teams were sometimes reluctant to share publicly because they wanted to protect intellectual property. In the years since their work though it has become straightforward to share documents with fine grained access control (see google documents, dropbox and/or github.)

Zoltowski and Oakes also recognize the importance of encouraging each cohort to lay the groundwork for the next: ``Finally, attention to transitioning to the next semester is explicitly part of the course schedule and discourse."


\section{The Course}

We call our course ``Autonomous Robotics Lab". It is offered every semester, initially to a maximum of 12 students. As mentioned, in order to provide continuity and a sense of a greater purpose, we've chosen a ``major project" objective which was compelling to students, could not be solved in one semester, and provided the kind of learning opportunities that we were after. The goal we set for ourselves (the ``Campus Rover'' robot) is a partially unsolved problem with plenty of juicy sub-problems to keep us occupied.

\subsection{Who is admitted} As an experimental course we are being highly selective of the (up to 12) students who can take this course. Being experimental means a curriculum that is fluid with incomplete scaffolding and teaching staff who is also still learning. We administer this as follows:

We admit only Juniors, Seniors and Graduate students. The catalog states that permission from the instructor is required. Students are asked to fill out a short (Google) survey asking about grades received in key Computer Science courses, as well as a statement of qualification stating in their own words what previous experiences prepare them for the course. Finally, they are individually interviewed over email to determine that they are strong programmers, disciplined, motivated and with a demonstrated track record.

\subsection{Course Schedule} Our semesters are 13 weeks in length. During the first 4 semesters, students explore basic robotics concepts and theories, and at the same time learn the fundamentals of ``ROS'' through lectures and programming assignments. This is a very rapid pace for the material being covered and may prove to be too rapid. 

During a transition week, the students form into teams of 2 to 4 and decide on projects that will span the remainder of the semester. Students are guided to certain projects are offered the opportunity to propose their own. The  requirements for projects at this stage are that the projects have a clear role in the overall goal, that they are achievable in the remainder of the semester, and that the team can get excited about them.

During the remaining 7 weeks, the students have to get organized around their project, decide on tasks and assignments, and work through to their final deliverable. During this period we continue to have weekly lab meetings, now to review progress, see weekly demonstrations, and propose deliverables for the next week.

\subsection{Learning Objectives} Given that each semester would be different we had to have a clear sense of our learning objectives. In a general sense, the goal is to expose students to the problem solving learning that comes into play when working on a large software project with many moving parts.

Our objectives fall into two categories, Robotics and Software Engineering. Here they are, edited for length:

\begin{itemize}
\item Understand fundamentals of mobile autonomous robotics, and the notions of localization, navigation and planning, including being able to explain the SLAM algorithms.
\item Demonstrate the ability to design, implement and test programs written for ROS (Robot Operating System.) 
\item Be able to explain as well as implement the key primitives of ROS: nodes, topics, commands and services.
\item Be effective working in teams, designing new algorithms and solving problems of navigation and robotics, brainstorming, collaborating, implementing, testing and demonstrating the results of their work.
\item Demonstrate professional and agile software engineering processes, including writing elegant, readable, documented code, working in rapid iterations, each with a goal and a demo, and performing weekly stand-up\cite{standup} meetings.
\end{itemize}

\section{Challenges}

\subsection{Technology} Robotics is an interesting sub-discipline of Computer Science, in that it is equal parts software and hardware, and equal parts theory and practice. In choosing what hardware and software platforms to adopt, we had the following priorities:
\begin{itemize}
    \item We didn't want the course to be viewed as ```playing with robots", so we wanted both hardware and software that was used in full scale commercial and research applications. 
    \item Our budget was limited, so it was important that the hardware we adopted was on the economical side of the scale.
    \item Our department is more software oriented than hardware oriented, so it was important that the hardware, while flexible, had excellent out-of-the-box performance, excellent documentation and support, and excellent customizability.
    \item Knowing the complexity of what we were attempting, we needed an operating system which was sophisticated, and that had large community support.
\end{itemize}
For the software platform, after considering many options, we decided on the Robot Operating System``ROS''\cite{ROS}. ROS is used extensively in both industry and research. The downside that we have encountered is that it is extremely complicated even with the help of a vibrant user community can be pretty tricky to get to work for our undergraduate students. For the hardware platform, again after review, we decided on the TurtleBot3\cite{Turtle} from Robotis. We are very comfortable with the choice of hardware. It is easy to use and reliable and quite economical.

\subsection{Driving Question} It is essential that a course using the MSMC model have an overarching ``driving question" which is a ``true north" for each cohort as it comes and goes. For us it was the ``Campus Rover'' (see above). 

This has worked out well, but with a caveat. Students joining the course often come with the expectation that they will complete and achieve the Campus Rover, which actually is not going to happen in the first set of runs of the program. We answer this challenge in several ways: First, we are careful to explain this ahead of time to the students to help them calibrate their expectations. During each semester we have different team ``sub'' projects, depending on the size of the group, their interests and abilities, and the status of the project. These subprojects are defined to be rewarding, instructional, and able to be completed by the end of the semester.

\subsection{Continuity}A major challenge is built-in to the ``multi-cohort" structure. Each semester brings a brand new group of students! It should be immediately obvious that at the end of the semester when the current cohort walks out the door, they take with them a lot of implicit knowledge. One way to summarize these challenges is of one of continuity. From the very beginning, during the ``recruiting" process, we emphasize the importance of continuity and how we plan to achieve that. Human nature can make this difficult - at the start of the semester, everyone is motivated jump in right away, and at the end of the semester, students are already thinking of what comes next. Below are our approaches to achieving continuity.

\subsubsection{Social Fabric}The first challenge is motivation. If we are to ask the students to help a future group of students we need to come up with ways to motivate them to do this. To that end, during the semester we try to engender a social connection between the current cohort and the preceding and following ones. We refer for example to the semesters as ``generations", e.g. ``gen0", ``gen1" and so on. We try to convey an appreciation of the work from the preceding group and a sense of obligation to the group that is coming next semester. To the extent that we are successful, this sets the stage for what comes next.

\subsubsection{Documentation}Especially during the last week, students are charged with documenting their achievements, including software, procedures, videos, check-lists, open issues, etc. Each team also writes a ``Letter to the next generation" where they describe at a global level what they want to make sure the next generation knows, and how they would best be able to continue the work. The challenge we've seen so far, is that generation n+1 gives short shrift to what was done by generation n.

\subsubsection{Teaching Assistant}We try to recruit next semester's teaching assistant from among this semester's cohort. This is a rather obvious way to create continuity and retention of ``institutional'' knowledge. One challenge is that, because we choose top students, the majority of them are Seniors or grad students who will not be around in the coming semester. A new refinement is to retain a part-time, but year-round grad student staff to help provide the continuity. This will afford better continuity and knowledge transfer.
\subsubsection{Lab Notebooks}Students were asked to maintain an online "Lab Notebook". Our instructions were that they should treat it as an experimental scientist would treat a lab notebook, making notes of what has been tried, tools used, failures, and results. Our motivation for this was that the notebooks would be a good tool to help the individual student keep track of their work. It would also serve as a useful tool for assessment and review. And finally it might give the student a nice artifact at the end of the course. In practice, this did not work very well, as it was difficult to motivate students to maintain their notebooks, and in the end they were chaotic and hard to reuse. We are no longer asking for a lab notebook.

\subsection{Assessment} Given the highly ambitious and even speculative nature of the main project, we do not assess on ``success of the project'', that is, whether students succeeded in building the Campus Rover. Rather, our assessments look at:

\subsubsection{Participation}
The course relies on students' ability to self-manage and self-organize. Students have to demonstrate their ability make progress without the professor or the curriculum dictating the exact steps they need to take. Participation therefore is more than just showing up for the weekly 3 hour meeting. It means showing up for your teammates as well. In discussions it is apparent that students really appreciate this freedom and responsibility - although not every student is ready for it.

\subsubsection{Demonstrations}
As mentioned, each week (after the first 3) teams need to put in writing what they will focus on for the coming week and what they will demonstrate during the following week. It is important to allow for unforeseen obstacles or delays, but with that in mind, the weekly demonstrations are very effective as a motivator for the students and as a wat to assess.

\subsubsection{Content}As mentioned, at the end of the semester we ask that the results of the work of each team be documented. We provide a general idea of what forms the documentation might take, but it is up to the students to digest the results of the semester and write them up in a way that makes sense to the next cohort.


\section{Progress to date}
As we enter the third semester of this program, we judge it a qualified success. In this final section we will review key features of the course design and our view of how well they worked, and suggestions for others meaning to implement a Multi-semester/Multi-cohort curriculum.
\subsection{What worked}
\begin{itemize}
\item BHAG: We feel that a having a overarching goal (BHAG) to frame the whole experience is very important. The characteristics we would look for are:
\begin{itemize}
\item A project which inspires and excites
\item A project which one can imagine is very hard, but not impossible
\item A project that students will want to talk about to friends and family
\end{itemize}

\item Robotics: Our students have demonstrated learning of advanced Robotics and Software Engineering concepts and skills. However as you see below, this learning did take valuable time out of the semester. Ideally there would be a preparatory course that would be a required pre-requisite.
\item Recruiting: While somewhat time consuming, our view is that it is essential to hand pick students. For us, requiring Juniors or above (although we have made exceptions) is a good starting point. However in addition we look for students who show great motivation, and history of persistence and inventiveness. We also give a frank preview of the likely challenges that they will encounter.
\item Lab: We have found that the lab “dynamic” is positive and energetic. Interestingly the cohort in one semester has shown a sense of obligation to those coming the next semester. We think this is a consequence of framing the individual semester within the context of the large overarching goal.
\item Outreach: We have been able to use the course as a platform for building and extending contacts with the robotics industry.
\item Team model: Dividing up the students into teams is important. And it is important to design team-objectives which are:
  \begin{itemize}
  \item Clearly supportive of the overarching goal (BHAG)
  \item Suitable for a team of approximately 3 students
  \item Self-contained so that the team can organize their own work and understand what their objectives are.
  \end{itemize}
Choosing the team objectives is the role of the instructor with input from the students. It is important to be able to explain to the students exactly what the team objective is, what it means, how it fits in the big picture, and suggestions on how it may be tackled. With that, the students are allowed organize themselves based on their interests and abilities.
\end{itemize}

\subsection{What we are still working on}
\begin{itemize}
\item Learning Curve: Learning the fundamentals of robotics (software and hardware) represents a very steep learning curve which means that we are not getting into the actual projects until 1/4 to 1/3 into the semester
\item Semester Deliverables: Until we achieve our BHAG, we need to figure out more satisfying “Semester Completion Deliverables” at the end of each semester.

\item Lab Notebook: As mentioned earlier, a feature of the course is the requirement that each student keep a ``lab notebook'' that they update with their progress, successes and setbacks, and notes to refer to later in the semester. Our idea was that this would serve as an assessment tool. This has not worked out. It was never updated, and when it was updated it felt like a ``make-work'' chore. However it has morphed into a true Lab Notebook, used, not for assessment, but as a way to help the student and student teams keep notes that they would use in future weeks.
\end{itemize}

\input{rpsnextsteps.tex}
\bibliographystyle{ACM-Reference-Format}
\bibliography{sample-bibliography}

\end{document}
