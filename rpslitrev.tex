\section{Review of Relevant Literature}

As far as we know we are the first to use the phrase "Multi-Semester, Multi-Cohort" so there will be no mention of it in the literature. But other than the phrase, there is certainly other relevant work that is quite instructive. 

One highly relevant example comes from ``Learning by Doing: Reflections of the Epics Program \cite{Epics}", Zoltowski and Oakes describe a ``multidisciplinary, vertically-integrated, student led, service-learning design course", where students of all levels join and participate in multi-semester service projects built around a central framework, called ``Epics". Here they describe in more detail how the service projects may span more than one semester or even academic year. This approach is analogous to our Multi-Semester, Multi-Cohort model, except that in our course there is but a single overarching objective. 

``However, in EPICS, the time lines are decoupled, so that projects can extend beyond one academic term. This allows us to scope projects to meet the needs of the community partner, not the requirements of the academic time line. It allows for iteration within the design process as problems and improvements are identified. In addition, students can participate in multiple semesters."

An interesting feature is that the courses count for one half the normal course credits, which allows students to take the course more than once. This could be used to help create continuity from semester to semester, as will be described below. 

Zoltowski and Oakes also mention the challenges of a multi-semester structure. They say that:  ``Multi-semester projects require structures different from most courses to allow transitioning and access to resources from one semester to another..." After which they describe the challenges of sharing reports from one semester to the next. Teams were sometimes reluctant to share publicly because they wanted to protect intellectual property. In the years since their work though it has become straightforward to share documents with fine grained access control (see google documents, dropbox and/or github.)

Zoltowski and Oakes also recognize the importance of encouraging each cohort to lay the groundwork for the next: ``Finally, attention to transitioning to the next semester is explicitly part of the course schedule and discourse."

