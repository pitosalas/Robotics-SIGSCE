\section{Review of Relevant Literature}

As far as we know we are the first to use the phrase "Multi-Semester, Multi-Cohort" so there will be no mention of it in the literature. But other than the phrase, there is certainly other relevant work that is quite instructive. 

One highly relevant example comes from ``Learning by Doing: Reflections of the Epics Program \cite{Epics}", Zoltowski and Oakes describe a ``multidisciplinary, vertically-integrated, student led, service-learning design course", where students of all levels join and participate in multi-semester service projects built around a central framework, called ``Epics". In this instance, students are working on projects on behalf of community partners, who have their own time-lines. As a result projects can exceed a single semester, and even span multiple semesters. This approach is analogous to our Multi-Semester, Multi-Cohort model, except that in our course there is but a single overarching objective. An interesting feature in the ``Epics" program is that the courses count for one half the normal course credits, which allows students to take the course more than once. This could be used to help create continuity from semester to semester, as will be described below. 

Zoltowski and Oakes also mention the challenges of a multi-semester structure. One that we also experienced was the question of how information is conveyed from one semester (and in our case, one cohort as well) to the next. For ``Epics" the issue seemed to revolve around the technical means of sharing reports. Teams were sometimes reluctant to share publicly because they wanted to protect intellectual property. In the years since their work though it has become straightforward to share documents with fine grained access control (see Google documents, Dropbox and/or Github.) Zoltowski and Oakes also recognize the importance of encouraging each cohort to lay the groundwork for the next: ``Finally, attention to transitioning to the next semester is explicitly part of the course schedule and discourse."

In ``Vertically Integrated Project (VIP) Programs: Multidisciplinary Projects with Homes in Any Discipline" \cite{VIP}, Cullers, Hughes and Llewelly review a series of VIP programs. Their definition of VIP programs has a lot in common with ours but their motivations are different --- although quite relevant to our program. VIP ``unites undergraduate education and faculty research in a team based context." There is a specific emphasis on multi-year participation by students. In addition there is a focus on cost-savings, scalability and sustainable. These ideas are compelling ones for us to consider introducing into our approach. One key difference is that VIP programs seem to be strongly focused on research - that is providing a structure for undergraduates to participate over multiple years in research programs, essentially receiving credit instead of stipends. The cited paper, in addition to introducing and defining the concept of the Vertically Integrated Project, also introduces the VIP Consortium and a series of programs and information about each one.

