\section{Challenges}

\subsection{Technology} Robotics is an interesting sub-discipline of Computer Science, in that it is equal parts software and hardware, and equal parts theory and practice. In choosing what hardware and software platforms to adopt, we had the following priorities:
\begin{itemize}
    \item We didn't want the course to be viewed as ```playing with robots", so we wanted both hardware and software that was used in full scale commercial and research applications. 
    \item Our budget was limited, so it was important that the hardware we adopted was on the economical side of the scale.
    \item Our department is more software oriented than hardware oriented, so it was important that the hardware, while flexible, had excellent out-of-the-box performance, excellent documentation and support, and excellent customizability.
    \item Knowing the complexity of what we were attempting, we needed an operating system which was sophisticated, and that had large community support.
\end{itemize}
For the software platform, after considering many options, we decided on the Robot Operating System``ROS''\cite{ROS}. ROS is used extensively in both industry and research. The downside that we have encountered is that it is extremely complicated even with the help of a vibrant user community can be pretty tricky to get to work for our undergraduate students. For the hardware platform, again after review, we decided on the TurtleBot3\cite{Turtle} from Robotis. We are very comfortable with the choice of hardware. It is easy to use and reliable and quite economical.

\subsection{Driving Question} It is essential that a course using the MSMC model have an overarching ``driving question" which is a ``true north" for each cohort as it comes and goes. For us it was the ``Campus Rover'' (see above). 

This has worked out well, but with a caveat. Students joining the course often come with the expectation that they will complete and achieve the Campus Rover, which actually is not going to happen in the first set of runs of the program. We answer this challenge in several ways: First, we are careful to explain this ahead of time to the students to help them calibrate their expectations. During each semester we have different team ``sub'' projects, depending on the size of the group, their interests and abilities, and the status of the project. These subprojects are defined to be rewarding, instructional, and able to be completed by the end of the semester.

\subsection{Continuity}A major challenge is built-in to the ``multi-cohort" structure. Each semester brings a brand new group of students! It should be immediately obvious that at the end of the semester when the current cohort walks out the door, they take with them a lot of implicit knowledge. One way to summarize these challenges is of one of continuity. From the very beginning, during the ``recruiting" process, we emphasize the importance of continuity and how we plan to achieve that. Human nature can make this difficult - at the start of the semester, everyone is motivated jump in right away, and at the end of the semester, students are already thinking of what comes next. Below are our approaches to achieving continuity.

\subsubsection{Social Fabric}The first challenge is motivation. If we are to ask the students to help a future group of students we need to come up with ways to motivate them to do this. To that end, during the semester we try to engender a social connection between the current cohort and the preceding and following ones. We refer for example to the semesters as ``generations", e.g. ``gen0", ``gen1" and so on. We try to convey an appreciation of the work from the preceding group and a sense of obligation to the group that is coming next semester. To the extent that we are successful, this sets the stage for what comes next.

\subsubsection{Documentation}Especially during the last week, students are charged with documenting their achievements, including software, procedures, videos, check-lists, open issues, etc. Each team also writes a ``Letter to the next generation" where they describe at a global level what they want to make sure the next generation knows, and how they would best be able to continue the work. The challenge we've seen so far, is that generation n+1 gives short shrift to what was done by generation n.

\subsubsection{Teaching Assistant}We try to recruit next semester's teaching assistant from among this semester's cohort. This is a rather obvious way to create continuity and retention of ``institutional'' knowledge. One challenge is that, because we choose top students, the majority of them are Seniors or grad students who will not be around in the coming semester. A new refinement is to retain a part-time, but year-round grad student staff to help provide the continuity. This will afford better continuity and knowledge transfer.
\subsubsection{Lab Notebooks}Students were asked to maintain an online "Lab Notebook". Our instructions were that they should treat it as an experimental scientist would treat a lab notebook, making notes of what has been tried, tools used, failures, and results. Our motivation for this was that the notebooks would be a good tool to help the individual student keep track of their work. It would also serve as a useful tool for assessment and review. And finally it might give the student a nice artifact at the end of the course. In practice, this did not work very well, as it was difficult to motivate students to maintain their notebooks, and in the end they were chaotic and hard to reuse. We are no longer asking for a lab notebook.

\subsection{Assessment} Given the highly ambitious and even speculative nature of the main project, we do not assess on ``success of the project'', that is, whether students succeeded in building the Campus Rover. Rather, our assessments look at:

\subsubsection{Participation}
The course relies on students' ability to self-manage and self-organize. Students have to demonstrate their ability make progress without the professor or the curriculum dictating the exact steps they need to take. Participation therefore is more than just showing up for the weekly 3 hour meeting. It means showing up for your teammates as well. In discussions it is apparent that students really appreciate this freedom and responsibility - although not every student is ready for it.

\subsubsection{Demonstrations}
As mentioned, each week (after the first 3) teams need to put in writing what they will focus on for the coming week and what they will demonstrate during the following week. It is important to allow for unforeseen obstacles or delays, but with that in mind, the weekly demonstrations are very effective as a motivator for the students and as a wat to assess.

\subsubsection{Content}As mentioned, at the end of the semester we ask that the results of the work of each team be documented. We provide a general idea of what forms the documentation might take, but it is up to the students to digest the results of the semester and write them up in a way that makes sense to the next cohort.

