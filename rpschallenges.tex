\section{Challenges}

\subsection{Technology} Robotics is an interesting discipline in that it is equal parts software and hardware, and equal parts theory and practice. Over two semesters so far, we’ve honed in on a software platform (``ROS" \cite{ROS}) and a hardware platform (``Turtlebot" \cite{Turtle}) that have the right characteristics needed as of now. These are both highly complex, and while appropriate, present a major learning curve for the students.

\subsection{Curriculum Design} The curriculum is built around the long term goal of the ``Campus Rover" (see above) and while students are not actually implementing that yet, it gives form and meaning to the curriculum. To support the intended learning objectives, the majority of the time is allotted to students work in teams. 

Each semester could have different team ``sub" missions, depending on the size of the group, their interests and abilities, and the status of the project. In the current semester, for example, the teams are: ``Architecture", ``Navigation" and ``New Platform".

\subsection{Continuity}A major challenge is built-in to the ``multi-cohort" structure. Each semester there are a brand new group of students! From the very beginning, during the ``recruiting" process, we emphasize the importance of continuity, of leaving the table ``set" for the next group coming next semester. We are exploring many possible ways of doing this, some we have implemented:
\subsubsection{Social Fabric}During the semester we try to engender a social connection between the current cohort and the preceding and following ones. We refer for example to the semesters as ``generations", e.g. ``gen0", ``gen1" and so on. We try to convey an appreciation of the work from the preceding group and a sense of obligation to the group that is coming next semester.
\subsubsection{Documentation}Especially during the last week, students are charged with documenting their achievements, including software, procedures, videos, check-lists, open issues, etc. Each team also writes a ``Letter to the next generation" where they describe at a global level what they want to make sure the next generation knows, and how they would best be able to continue the work.
\subsubsection{Teaching Assistant}We try to recruit next semester’s teaching assistant from among this semester’s cohort. Because we choose top students, the majority of them are Seniors or grad students who will not be around in the coming semester. With funding, we are considering having a part-time, but year-round grad student staff to help provide the continuity.
\subsubsection{Lab Notebooks}Students were asked to maintain an online "Lab Notebook" with the expectation and intent that these would be useful both for assessment and for continuity. In practice, this did not work very well, as it was difficult to motivate students to maintain their notebooks, and in the end they were chaotic and hard to reuse.
\subsection{Assessment} Given the highly ambitious and even speculative nature of the main project, we do not assess on ``success of the project", that is, whether students succeeded in building the Campus Rover. Rather, our assessments look at:

\subsubsection{Participation}
The course relies on students' ability to self-manage and self-organize. Students have to demonstrate their ability make progress without the professor or the curriculum dictating the exact steps they need to take. Participation therefore is more than just showing up for the weekly 3 hour meeting. It means showing up for your teammates as well. In discussions it is apparent that students really appreciate this freedom and responsibility - although not every student is ready for it.

\subsubsection{Demonstrations}
As mentioned, each week (after the first 3) teams need to put in writing what they will focus on for the coming week and what they will demonstrate during the following week. Students have to learn and then be motivated to think in terms of concrete weekly objectives, followed by a demonstration to the rest of the class. We track this during the semester.

\subsubsection{Content}As mentioned, at the end of the semester we ask that the results of the work of each team be documented. We provide a general idea of what forms the documentation might take, but it is up to the students to digest the results of the semester and write them up in a way that makes sense to the next cohort.

