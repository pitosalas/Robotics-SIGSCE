\section{Progress to date}
As we enter the third semester of this program, we judge it a qualified success. In this final section we will review key features of the course design and our view of how well they worked, and suggestions for others meaning to implement a Multi-semester/Multi-cohort curriculum.
\subsection{What worked}
\begin{itemize}
\item BHAG: We feel that a having a overarching goal (BHAG) to frame the whole experience is very important. The characteristics we would look for are:
\begin{itemize}
\item A project which inspires and excites
\item A project which one can imagine is very hard, but not impossible
\item A project that students will want to talk about to friends and family
\end{itemize}

\item Robotics: Our students have demonstrated learning of advanced Robotics and Software Engineering concepts and skills. However as you see below, this learning did take valuable time out of the semester. Ideally there would be a preparatory course that would be a required pre-requisite.
\item Recruiting: While somewhat time consuming, our view is that it is essential to hand pick students. For us, requiring Juniors or above (although we have made exceptions) is a good starting point. However in addition we look for students who show great motivation, and history of persistence and inventiveness. We also give a frank preview of the likely challenges that they will encounter.
\item Lab: We have found that the lab “dynamic” is positive and energetic. Interestingly the cohort in one semester has shown a sense of obligation to those coming the next semester. We think this is a consequence of framing the individual semester within the context of the large overarching goal.
\item Outreach: We have been able to use the course as a platform for building and extending contacts with the robotics industry.
\item Team model: Dividing up the students into teams is important. And it is important to design team-objectives which are:
  \begin{itemize}
  \item Clearly supportive of the overarching goal (BHAG)
  \item Suitable for a team of approximately 3 students
  \item Self-contained so that the team can organize their own work and understand what their objectives are.
  \end{itemize}
Choosing the team objectives is the role of the instructor with input from the students. It is important to be able to explain to the students exactly what the team objective is, what it means, how it fits in the big picture, and suggestions on how it may be tackled. With that, the students are allowed organize themselves based on their interests and abilities.
\end{itemize}

\subsection{What we are still working on}
\begin{itemize}
\item Learning Curve: Learning the fundamentals of robotics (software and hardware) represents a very steep learning curve which means that we are not getting into the actual projects until 1/4 to 1/3 into the semester
\item Semester Deliverables: Until we achieve our BHAG, we need to figure out more satisfying “Semester Completion Deliverables” at the end of each semester.

\item Lab Notebook: As mentioned earlier, a feature of the course is the requirement that each student keep a ``lab notebook'' that they update with their progress, successes and setbacks, and notes to refer to later in the semester. Our idea was that this would serve as an assessment tool. This has not worked out. It was never updated, and when it was updated it felt like a ``make-work'' chore. However it has morphed into a true Lab Notebook, used, not for assessment, but as a way to help the student and student teams keep notes that they would use in future weeks.
\end{itemize}
