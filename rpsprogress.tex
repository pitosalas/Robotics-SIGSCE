\section{Progress to date}
As we enter the third semester of this program, we judge it a qualified success. In this final section we will review key features of the course design and our view of how well they worked, and suggestions for others meaning to implement a Multi-semester/Multi-cohort curriculum.
\subsection{What worked}
\subsubsection{Driving Question}
We feel that having an overarching goal (driving question) to frame the whole experience is very important. The characteristics we would look for are:
\begin{itemize}
\item A goal which inspires and excites
\item A goal which is very hard, but not impossible
\item A goal that students would be proud to share with others
\end{itemize}

\subsubsection{Robotics} Our students have demonstrated learning of advanced Robotics and Software Engineering concepts and skills. However as you see below, this learning did take valuable time out of the semester. Ideally there would be a preparatory course that would be a required prerequisite.
\subsubsection{Selection}While somewhat time consuming, our view is that it is essential to hand pick students. We know that just looking at grades is not enough.  Our criteria and motivations are:
\begin{itemize}
\item Students will be using and reading a lot of advanced source code, in an advanced environment so we are looking for students who have demonstrated an ability to work on larger software projects, and who seem to really enjoy programming.
\item The tools we use, and the advanced algorithms that we have to use, never work as documented or out of the box. This is not surprising of course. We are looking for students with perseverance in the face of obstacles, with a desire to get to the bottom of things, who are comfortable asking questions in person or online, and who don't get overly frustrated.
\item Once we complete the initial 3 weeks of learning the software platform (ROS) students form teams (see above) to work on sub-projects. Those teams are mostly self-managing, with members making commitments to each other and being able to figure out a path on their own. While the teaching staff always present, we are looking for students who are ready for the unstructured way real-world assignments and projects work.
\end{itemize}
With such a high bar, one might fairly ask how to select for such students. Certainly simply looking at grades isn't enough. Our approach has been as follows: 
\begin{itemize}
\item The course description makes clear that the course will be time consuming and that we are looking for people like those described above.
\item The course is only open to Juniors, Seniors and grad students (although exceptions can and have been made).
\item A questionnaire asks for grades achieved for key other courses, for a resume, and for a brief statement
\item Finally a quick interview tries to screen for the above requirements.  We also give a frank preview of the likely challenges that they will encounter.
\end{itemize}
\subsubsection{Lab Space} We have found that the lab ``dynamic'' is positive and energetic. Interestingly the cohort in one semester has shown a sense of obligation to those coming the next semester. We think this is a consequence of framing the individual semester within the context of the large overarching goal.
\subsubsection{Outreach} We have been able to use the course as a platform for building and extending contacts with the robotics industry. We think it is because the nature of the work and expectations of students taking this course maps well onto what robotics companies are looking for.
\subsubsection{Team model}Dividing up the students into teams is important (see above.) And it is important to design team-objectives which are:
\begin{itemize}
\item Clearly supportive of the overarching goal.
\item Suitable for a team of approximately 3 students
\item Self-contained so that the team can organize their own work and understand what their objectives are.
\end{itemize}
Choosing the team objectives is the role of the instructor with input from the students. It is important to be able to explain to the students exactly what the team objective is, what it means, how it fits in the big picture, and offer suggestions on how it may be tackled. With that, the students are allowed organize themselves based on their interests and abilities.

\subsection{What we are still working on}
\subsubsection{Learning Curve}Learning the fundamentals of robotics (software and hardware) represents a very steep learning curve. Previously we let the students tackle that learning curve through self study. The effect of this was that we were not getting into the actual projects until week 4 or 5 of a 13 week semester. Our response to this is that we've designed a tight 3-4 week curriculum, more formally taking students through the foundations of the software and hardware platform. We hope that this will tighten up the early start.
\subsubsection{Semester Deliverables}Until we achieve our BHAG, we need to figure out more satisfying ``Semester Completion Deliverables'' at the end of each semester. What we will do now is to pick the sub-projects with a more specific focus on what the deliverables would be, making sure that they are achievable and rewarding.
\subsubsection{Lab Notebook}As mentioned earlier, a feature of the course was the requirement that each student keep a ``lab notebook'' that they update with their progress, successes and setbacks, and notes to refer to later in the semester. Our idea was that this would serve as an assessment tool. This has not worked out. It was never updated, and when it was updated it felt like a ``make-work'' chore. However it has morphed into a true Lab Notebook, used, not for assessment, but as a way to help the student and student teams keep notes that they would use in future weeks.
\subsubsection{New Courses}We are seriously considering additional courses. While having a dedicated course for Autonomous Robotics Lab is valuable, one bureaucratic obstacle is that a student cannot take the same course twice, so one who wanted to continue on their project would be blocked. 

Instead of running this Lab course every semester, we could alternate it with an independent study semester where students who elected to could continue their work in robotics as an independent study. A separate approach to dealing with the Learning Curve problem described above, is to design a new course, e.g. "Introduction to Autonomous Robotics", as a conventional project-based course, offer it once a year, and make it a prerequisite to the MSMC course described in this paper.

